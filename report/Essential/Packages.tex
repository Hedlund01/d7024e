%-----------------------------------------------
%        Hugos Template - V. 2021.04.09
%-----------------------------------------------
%                   Settings
%-----------------------------------------------
%   Ändra språk, och snabb compile
\usepackage[english, provide=*]{babel}       % Språk. [swedish] eller [english].
\usepackage[]{graphicx}           % Bilder. [demo] för snabb compile!

%   Extra
%\usepackage{mhchem}    % Kemiekvationer.
%\usepackage{changepage}           % Lokal ändring i margin, \begin{adjustwidth}{-1cm}{-1cm}


%-----------------------------------------------
%                   Packages
%-----------------------------------------------
\usepackage[utf8]{inputenc}          % Kodning av text.
\usepackage[]{biblatex}              % (Vancouver), \cite \textcite
%..[style=authoryear-ibid]           % (Harvard), \parencite \textcite
%..[style=apa]                       % (APA) \textcite \parencite
%\DeclareLanguageMapping{swedish}{swedish-apa}


\pagenumbering{gobble}       % Stoppar sidnumrering på titelsida.
\usepackage{mathtools}       % Ekvationer.
\usepackage{bm}              % Bold matte
\usepackage[]{geometry}      % Sidlayout.
\usepackage{float}           % För [H].
\usepackage[T1]{fontenc}     % Bra för åäö.
\usepackage{csquotes}        % Fancy Quotes
\usepackage[expansion=false]{microtype}       % Fixar bättre textlayout.

\usepackage[margin=15pt,font={small},labelfont={bf}%
,justification=centering]{caption}      % Snygg figurtext.
\usepackage{booktabs}                   % Snygga Tabeller.
\usepackage{longtable}                  % Långa tabeller, flera sidor.
\setlength{\abovecaptionskip}{3pt}      % Flyttar tabelltext närmare.
\renewcommand{\arraystretch}{1.2}       % Gör tabeller  större.
\setlength{\parindent}{0cm}             % Tar bort "indent" på ny rad.
\usepackage[none]{hyphenat}             % Slutar dela upp ord.

\bibliography{Essential/References.bib} % Mapp för Källor.
\graphicspath{{Images/}}                % Mapp för bilder.
\usepackage{pgffor}                     % Loop för bilagor.
\usepackage{pdfpages}                   % PDFs.
\usepackage{tikz}                       % TikZ diagrams.
\usetikzlibrary{positioning, arrows.meta, shapes}  % TikZ libraries.
\addto{\captionsswedish}{\renewcommand*%%
{\contentsname}{Innehållsförteckning}}  % Ändrar namn -> Innehållsförteckning

\usepackage[colorlinks]{hyperref}                  % Hyperlänkar + färg.
\usepackage[nameinlink,noabbrev,swedish]{cleveref} % \cref{} & \Cref{}.
\usepackage{color}                                 % Färg.
\definecolor{royalblue}{rgb}{0.0, 0.14, 0.4}       % Egen färg.
\hypersetup{                                       % Färg på hyperlänkar.
     colorlinks  = true,
     linkcolor   = black,              % internal links.
     citecolor   = royalblue,          % bibliography.
     filecolor   = royalblue,          % file links.
     urlcolor    = royalblue           % external links.
}

% \usepackage{matlab-prettifier}          % Matlab text!
\usepackage[per-mode=symbol]{siunitx}   % si. Skriva enheter/nummer.
\sisetup{output-decimal-marker={,},%    % si. ("{,}" till "{.}" för engelska).
range-phrase=--,range-units=single,exponent-product=\cdot}

%-----------------------------------------------
%                 New commands                
%-----------------------------------------------
% Använd för radbyte. \n
\newcommand{\n}{\vskip 1em}

% Bold matte och text (Automatisk) ex. \B{TEXT och } $\B{\alpha}$
\newcommand*{\B}[1]{\ifmmode\bm{#1}\else\textbf{#1}\fi}

% Partial Derivative. \pde{a}{b}
\newcommand{\pde}[2]{\frac{\partial #1}{\partial #2}}

% Använd för matlab fil.
\newcommand{\matlabfile}[1]
{\UseRawInputEncoding\hyphenpenalty=50\exhyphenpenalty=50\lstinputlisting[style = Matlab-editor,basicstyle = \mlttfamily]{#1}}

% Använd för matlab text.
% \lstnewenvironment{matlab}
% {\UseRawInputEncoding\hyphenpenalty=50\exhyphenpenalty=50\lstset{style = Matlab-editor,basicstyle = \mlttfamily}}{}

\usepackage[page,toc,titletoc,title]{appendix}

% Kollar om det finns .tex fil Från Bilaga_1 till Bilaga_15.
% Genererar sida om det finns. Öka från 15 vid behov.
\edef\bilagamax{15}
\newcommand{\bilagaloop}{\foreach \i in {1, 2, 3, ...,\bilagamax} {
    \edef\FileName{Sections/Bilagor/Bilaga_\i}%
    \IfFileExists{\FileName.tex}{%
        \newpage%
        \section{}%
        \input{\FileName.tex}%
    }{}
}\IfFileExists{Sections/Bilagor/Bilaga_\bilagamax.tex} {\errmessage{MAX ANTAL BILAGOR, ÖKA "bilagamax" i Packages.tex}}
}

%%%%%%%%%%%%%%%%%%%%%%%%%%%%%%%%%%%%%%%%%%%%%%%%%%%%
%%  Creative Commons CC BY 4.0, Hugo Laestander %%%%
%%%%%%%%%%%%%%%%%%%%%%%%%%%%%%%%%%%%%%%%%%%%%%%%%%%%